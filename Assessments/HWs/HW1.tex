\documentclass[12pt]{article}

% Solutions toggle
\newif\ifsolutions
\solutionsfalse
%% \solutionstrue

% ASSIGNMENT NUMBER
\newcommand{\hwnumber}{1}
\newcommand{\booksection}{Dimensional Analysis}
\newcommand{\duedate}{}
% -------------

%%% Packages
\usepackage[margin=1in, footskip=24pt, headheight=24pt]{geometry}
\usepackage{amsmath, amssymb, amsthm, graphicx}
\usepackage{mathtools}
\DeclarePairedDelimiter\ceil{\lceil}{\rceil}
\DeclarePairedDelimiter\floor{\lfloor}{\rfloor}
\usepackage[colorlinks, urlcolor=blue]{hyperref}
\usepackage{color}
\usepackage{comment}
\usepackage{enumerate}
\usepackage{lastpage}
\usepackage{multirow, multicol}
\usepackage{tikz}
\usetikzlibrary{matrix,decorations.text,decorations.pathmorphing,decorations.markings,arrows,calc,shapes.geometric,patterns,shadows,intersections,decorations.markings,decorations.pathreplacing,decorations.pathreplacing,backgrounds,angles,quotes}
\usepackage{pgfplots}
\pgfplotsset{compat=1.16}

\usepackage{fancyhdr}

\pagestyle{fancy}
%% \renewcommand{\familydefault}{\sfdefault}

\newcommand{\R}{\mathbb{R}}
\newcommand{\ddx}{\frac{d}{dx}}

\global\long\def\V#1{\boldsymbol{#1}} %vector
\global\long\def\M#1{\boldsymbol{#1}} %matrix

\global\long\def\D#1{\Delta#1} %\D{t} for time step size
\global\long\def\d#1{\delta#1} %\d{t} for small increment

\global\long\def\norm#1{\left\Vert #1\right\Vert }
\global\long\def\abs#1{\left|#1\right|}

\global\long\def\grad{\M{\nabla}}
\global\long\def\av#1{\left\langle #1\right\rangle }

% HEADER MACROS
\newcommand{\term}{Spring 2022 \& 2023}
\newcommand{\coursename}{Intro Math Modeling}
\newcommand{\coursenumber}{MATH-UA 251}
\newcommand{\course}{\coursename \ (\coursenumber)}

\fancyhead[RO]{\term}
\fancyhead[LO]{\course}
% -------------

%%% Theorem Styles
\theoremstyle{definition}
\newtheorem{ex}{Exercise}

%%%%%%%%%%%%%%%%%%%%%%%% Solutions %%%%%%%%%%%%%%%%%%%%%%%%%%%%%
% \begin{solution} and \begin{answerspace} must be at the beginning of the line.
% Doesn't work inside the \myversions command. Use if statements instead.
% No underscores in comment names

\ifsolutions
\newenvironment{solution}{\color{blue}}{} \excludecomment{answerspace} \newenvironment{notes}{\color{red} \noindent Grading Notes:}{}
\else
\excludecomment{notes} \excludecomment{solution} \includecomment{answerspace} 
\fi
%%%%%%%%%%%%%%%%%%%% End Solutions %%%%%%%%%%%%%%%%%%%%%%%e}


\begin{document}
% HEADER
\begin{center}
%% \ifsolutions
%%   \textbf{\Large Homework \hwnumber\ - \booksection\ (Solutions)}\\
%% \else
%%   \textbf{\Large Homework \hwnumber\ - \booksection}\\
%% \fi
\ifsolutions
  \textbf{\Large Homework \hwnumber\ (Solutions)}\\
\else
  \textbf{\Large Homework \hwnumber}\\
\fi
\vspace{12pt}
Due date: someday, sometime! \duedate

Submit on NYU Brightspace.
\end{center}

%% \noindent Please give complete, well-written solutions to the following exercise. Provide sufficient justification and explanation for a classmate who has not worked on the exercise to understand your solution.


% -------------------

\begin{ex}

  [50 pts] We wish to use the Buckingham $\pi$ theorem to estimate the drag force on an airplane cruising at $400\;\mathrm{km/hr}$ in standard air (atmospheric pressure) using a scaled $1:10$ down model (geometrical dimensions of the model are $0.1$ of that of the prototype). Let the air speed in the wind tunnel be $400\;\mathrm{km/hr}$ as well. Assuming the same air temperature for the model and prototype, determine the required pressure in the tunnel, and the drag on the prototype corresponding to $1\;\mathrm{N}$ on the model.

\vspace{0.5cm}\noindent\textbf{Hints:}
\begin{enumerate}[(i)]
  \item The drag force, $F$, is expected to depend on the following parameters: $V$ (wind speed), $\mu$ (viscosity), $\rho$ (density), $\ell$ (length of the airplane), and $w$ (wing span of the airplane).
  \item Assume that the air viscosity does not depend on the air pressure.
  \item Use Ideal Gas Law $p MW=\rho R T$ to relate the air density to pressure. ($p$: pressure, $R$: universal gas constant, $MW$: molecular weight of air (constant), $T$: absolute temperature)
\end{enumerate}
		
\begin{solution}
  We have $F=f(V,\mu,\rho,\ell,w)$. We use the Buckingham $\pi$ theorem to find the dimensionless group.
  \begin{align*}
    F&\;\dot{=}\;MLT^{-2},\\
    V&\;\dot{=}\;LT^{-1},\\
    \mu&\;\dot{=}\;ML^{-1}T^{-1},\\
    \rho&\;\dot{=}\;ML^{-3},\\
    \ell&\;\dot{=}\;L,\\
    w&\;\dot{=}\;L.
  \end{align*}
  Therefore, the number of reference dimensions is $r=3$, which, by Buckingham $\pi$ theorem, yields $6-3=3$ dimensionless groups. We pick $\ell$, $V$, and $\rho$ as the repeating variables:
  \begin{align*}
    &\Pi_1=F\ell^aV^b\rho^c=M^0L^0T^0\Rightarrow a=-2,b=-2,c=-1\Rightarrow \Pi_1=\frac{F}{\ell^2V^2\rho},\\
    &\Pi_2=\mu\ell^aV^b\rho^c=M^0L^0T^0\Rightarrow a=-1,b=-1,c=-1\Rightarrow \Pi_2=\frac{\mu}{\ell V\rho},\\
    &\Pi_3=w\ell^aV^b\rho^c=M^0L^0T^0\Rightarrow a=-1,b=0,c=0\Rightarrow \Pi_3=\frac{\ell}{w}.
  \end{align*}
  Finally, we can express the results of the dimensional analysis as
  \begin{equation*}
    \Pi_1=\phi\left(\Pi_2,\Pi_3\right)\Rightarrow\Pi_1=\psi\left(1/\Pi_2,\Pi_3\right)\Rightarrow\frac{F}{\ell^2V^2\rho}=\psi\left(\frac{\ell V\rho}{\mu},\frac{\ell}{w}\right).
  \end{equation*}

  Now, for the model to represent the prototype, we need
  \begin{equation*}
    \frac{\ell_m V_m\rho_m}{\mu_m}=\frac{\ell V\rho}{\mu},\quad \frac{\ell}{w}=\frac{\ell_m}{w_m},
  \end{equation*}
  where the subscript $m$ denotes the model. The second condition is automatically satisfied since the model is a $1:10$ scaled down version of the prototype: $\ell_m=0.1\ell$ and $w_m=0.1w$. Consequently, we only need to ensure that the Reynolds numbers are the same:
  \begin{equation*}
    \frac{\ell_m V_m\rho_m}{\mu_m}=\frac{\ell V\rho}{\mu}\Rightarrow\frac{\rho_m}{\rho}=\frac{\mu_m}{\mu}\frac{V}{V_m}\frac{\ell}{\ell_m}=\frac{\mu_m}{\mu}\cdot 1\cdot 10=10\frac{\mu_m}{\mu}.
  \end{equation*}
  As this result indicates, the same fluid with $\rho_m=\rho$ and $\mu_m=\mu$ cannot be used if the Reynolds number similarity is to be maintained. We can pressurize the wind tunnel to increase the density $\rho_m$. We assume that $\mu_m$ does not change with pressure significantly. Therefore, $\mu_m=\mu$, and we need
  \begin{equation*}
    \frac{\rho_m}{\rho}=10\Rightarrow \frac{p_m MW/RT_m}{p MW/RT}=\frac{p_m}{p}=10.
  \end{equation*}
  So, the pressure needed to ensure that $\mathrm{Re}_m=\mathrm{Re}$ is $10$ times higher than the pressure at which the actual prototype operates. Given that $p=1\;\mathrm{atm}$, $p_m=10\;\mathrm{atm}$.

  Now that $\Pi_{1m}=\Pi_{1}$ and $\Pi_{2m}=\Pi_2$, we conclude that
  \begin{equation*}
    \frac{F}{\ell^2V^2\rho}=\frac{F_m}{\ell_m^2V_m^2\rho_m}\Rightarrow F=\left(\frac{\ell}{\ell_m}\right)^2\left(\frac{V}{V_m}\right)^2\left(\frac{\rho}{\rho_m}\right)F_m=10^2\cdot 1\cdot 0.1F_m=10F_m.
  \end{equation*}
  So for a drag of $F_m=1\;\mathrm{N}$ on the model, the corresponding drag on the prototype is $F=10\;\mathrm{N}$.
\end{solution}
\end{ex}

\begin{ex}

  [50 pts] From dimensional analysis, we know that the drag coefficient $C_D$ of a smooth sphere moving in a fluid is a function of Reynolds number:
  \begin{equation*}
    C_D=f(\mathrm{Re})=f\left(\frac{\rho V D}{\mu}\right),
  \end{equation*}
  where $\rho$ and $\mu$ are the density and viscosity of the fluid (e.g., air), $D$ is the diameter of the sphere, and $V$ is its velocity. The drag force can then be expressed as
  \begin{equation*}
    F=\tfrac{1}{2}C_D\rho A V^2
  \end{equation*}
  with $A$ as the projected area. Note that you can derive the Stokes drag force formula (valid when $\mathrm{Re}\lesssim 1$) from the above equation by inserting $C_D=24/\mathrm{Re}$.

  Now consider two raindrops, one with diameter $2\;\mathrm{mm}$ and another one with diameter $4\;\mathrm{mm}$. Which one falls faster in the air and by how much?

\vspace{0.5cm}\noindent\textbf{Hints:}
\begin{enumerate}[(i)]
\item Assume that the drops fall at their `terminal' velocities, which is the velocity at which the drag force and gravitational forces cancel each other: $F=\tfrac{1}{6}\pi D^3(\rho_s-\rho)g$, where $\tfrac{1}{6}\pi D^3$ is the volume of the droplet, $\rho_s$ is its density ($\approx 1000\;\mathrm{kg/m^3}$), and $g\approx 9.8\;\mathrm{m/s^2}$ is the gravitational acceleration. Also, for the air, use $\rho=1\;\mathrm{kg/m^3}$ and $\mu=2\times 10^{-5}\;\mathrm{Pa\cdot s}$.
\item Initially assume that the drops fall in the Stokes regime (i.e., $\mathrm{Re}\lesssim 1$) and use the Stokes drag formula. After finding the terminal velocity calculate the $\mathrm{Re}$ number to verify your initial assumption.
\item In case the Stokes regime is not valid: for high Reynolds number $10^3\lesssim\mathrm{Re}\lesssim10^5$, one can assume that $C_D$ is constant and not a function of $\mathrm{Re}$. Use $C_D\approx 0.4$ and redo the above process. Verify if the Reynolds number is within the range.
\end{enumerate}

\begin{solution}
  We first derive the Stokes drag formula:
  \begin{equation*}
    F=\tfrac{1}{2}C_D\rho A V^2=\tfrac{1}{2}\left(\tfrac{24\mu}{\rho V D}\right)\rho \left(\tfrac{1}{4}\pi D^2\right) V^2=3\pi\mu DV.
  \end{equation*}
  Now, we find the terminal velocity by balancing the Stokes drag and gravitational forces:
  \begin{equation*}
    3\pi\mu DV=\tfrac{1}{6}\pi D^3(\rho_s-\rho)g\Rightarrow V=\frac{D^2(\rho_s-\rho)g}{18\mu}.
  \end{equation*}
  We see that the terminal velocity is proportional to the square of the diameter. Therefore,
  \begin{equation*}
    \frac{V(D=4\;\mathrm{mm})}{V(D=2\;\mathrm{mm})}=2^2=4.
  \end{equation*}
  In other words, the $4\;\mathrm{mm}$-droplet falls $4$ times faster than the $2\;\mathrm{mm}$-droplet. We now need to verify that we could use the Stokes drag formula by calculating the Reynolds number:
  \begin{equation*}
    \mathrm{Re}=\frac{\rho VD}{\mu}=\frac{\rho(\rho_s-\rho)D^3 g}{18\mu^2}.
  \end{equation*}
  Using $\rho_s=1000\;\mathrm{kg/m^3},\rho=1\;\mathrm{kg/m^3},\mu=2\times 10^{-5}\;\mathrm{Pa\cdot s}$, the Reynolds number become $10885$ and $87082$ for $D=2\;\mathrm{mm}$ and $4\;\mathrm{mm}$, respectively. This is \textbf{very} high! So the Stokes law is not valid.

  Now we let $C_D=0.4$ and constant:
  \begin{equation*}
    \tfrac{1}{2}C_D\rho AV^2=\tfrac{1}{6}\pi D^3(\rho_s-\rho)g\Rightarrow V=\sqrt{\frac{4D(\rho_s-\rho)g}{3C_D\rho}}.
  \end{equation*}
  Therefore, the falling velocity is proportional to the square of the diameter:
\begin{equation*}
    \frac{V(D=4\;\mathrm{mm})}{V(D=2\;\mathrm{mm})}=\sqrt{2}=1.41.
  \end{equation*}
Using the new formula for the velocity to calculate the Reynolds number yields $\mathrm{Re}\approx 808$ and $\approx 2285$ for $D=2\;\mathrm{mm}$ and $4\;\mathrm{mm}$, respectively. This is approximately within the range for which $C_D$ remains constant. Finally, we find the falling velocities as $8.08\;\mathrm{m/s}$ and $11.43\;\mathrm{m/s}$ for the $2$-$\mathrm{mm}$ and $4$-$\mathrm{mm}$ raindrops, respectively.
\end{solution}
\end{ex}

\begin{ex}

  [10+10+10+10 pts] Heat transfer flux from a surface at temperature $T_s$ to an environment at temperature $T_{\infty}$ is given by $h(T_s-T_{\infty})$, where $h$ is the heat transfer coefficient with dimensions $\mathrm{energy/(time\cdot area\cdot temperature)}$. For a sphere suspended in a still air, the heat transfer coefficient depends on the sphere diameter, $D$, and conductivity of the air, $k$, (with dimensions $\mathrm{energy/(time\cdot length\cdot temperature)}$). Hence, one can write $h=f(k,D)$.

\begin{enumerate}[(i)]
\item List all of the parameters and their dimensions. Hint: $\mathrm{energy}\doteq\mathrm{force\cdot length}$. 
\item Find the number of \emph{reference} dimensions, and subsequently, the number of dimensionless groups.
\item Find the dimensionless groups using the method described in class.
\item Interpret the result.
\end{enumerate}
  
\begin{solution}
  We use $M,L,T,\theta$ to denote the primary dimensions mass, length, time, and temperature.
  \begin{align*}
    h&\doteq\mathrm{\frac{energy}{time\cdot area\cdot temperature}}\doteq\mathrm{\frac{force\cdot length}{time\cdot area\cdot temperature}}\doteq\frac{MLT^{-2}L}{TL^2\theta}\doteq MT^{-3}\theta^{-1},\\
    k&\doteq\mathrm{\frac{energy}{time\cdot length\cdot temperature}}\doteq\mathrm{\frac{force\cdot length}{time\cdot length\cdot temperature}}\doteq\frac{MLT^{-2}L}{TL\theta}\doteq MLT^{-3}\theta^{-1},\\
    D&\doteq L.
  \end{align*}

  A simple inspection shows that $h$ and $k$ differ from each other by a length scale. Indeed, the dimensions required to fully describe the system are $MLT^{-3}\theta^{-1}$ and $L$. Hence, the total number of dimensionless group(s) is $3-2=1$. We can take $D$ and $k$ as the repeating parameters and write the dimensionless group as
  $$\Pi=h k^a D^b$$
  It is easy to find that we need $a=-1,b=1$ for $\Pi$ to be dimensionless, i.e., $\Pi=M^0L^0T^0\theta^0$. Since the problem is characterized by only one dimensionless number, we conclude that
  $$\Pi=\frac{hD}{k}=\mathrm{constant}.$$

  The dimensionless group $h\ell/k$ is known as the Nusselt number, where $\ell$ is a length scale ($D$ for a sphere). For a sphere in still air, one can show (by solving transport equations) that $hD/k=2$. 
\end{solution}  

\end{ex}

\end{document}
