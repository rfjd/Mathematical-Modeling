\documentclass[amsmath,amsfonts,amssymb,letterpaper,10pt]{article}
\usepackage[tmargin=0.5in,bmargin=0.5in,lmargin=0.75in,rmargin=0.75in]{geometry}
\usepackage{pgfplots}
\usepackage{enumitem}
\usepackage{tabularx}
\usepackage{booktabs}
\setlength\parindent{0pt}
\pagenumbering{gobble}

\title{\vspace{-1cm}\Large{\textbf{Intro Math Modeling, NYU Courant (Aref Hashemi)}}}
\author{}
\date{}

\begin{document}
\maketitle

\vspace{-1cm}
\noindent{\large\textbf{Description}}\\
In this course, we mathematically model and analyze a variety of physical and biological systems. The tools include calculus, algebra, probability, ordinary and partial differential equations, numerical analysis, and stochastic processes. Whenever needed, we use python for coding.

\vspace{0.25cm}
\noindent{\large\textbf{Recommended Textbooks}}\\
\vspace{-0.5cm}
\begin{itemize}[leftmargin=*]\setlength\itemsep{-0.3em}
\item \emph{Nonlinear Dynamics \& Chaos} by Steven H. Strogatz
\item \emph{Transport Phenomena} by R. Byron Bird, Warren E. Stewart, and Edwin N. Lightfoot
\item \emph{An Introduction to Mechanics} by Daniel Kleppner and Robert Kolenkow
\item \emph{Fundamentals of Fluid Mechanics} by Bruce R. Munson, Donald F. Young, and Theodore H. Okiishi
\item \emph{Heat Transfer} by J. P. Holman
\item \emph{Numerical Analysis for Engineers and Scientists} by G. H. Miller
\end{itemize}

\noindent{\large\textbf{Tentative Schedule}}\\
\vspace{-0.75cm}
\begin{center}
  \begin{tabular}{m{0.11\textwidth}m{0.62\textwidth}m{0.22\textwidth}}
    \toprule
    {\large{\textbf{Week}}} & {\large{\textbf{Topics}}} & {\large{\textbf{Reading}}}\\
    \toprule
    1 (01/23) & dimensional analysis, Buckingham $\pi$ theorem, coordinate systems, shell balance & {\shortstack[l]{Munson Ch1 \& Ch7\\Holman Ch1}}\\ [1ex]
    \hline
    2 (01/30) & differential form of transport equations, derivation of continuity and heat equations in Cartesian and Cylindrical coordinates, vector formulation & {\shortstack[l]{Holman Ch1\\Bird Ch11}}\\ [1ex]
    \hline
    3 (02/06) & modeling of transport equations, 1D transport systems, different types of BCs (Dirichlet, Neumann, Robin), conduction-convection problems (heat fin), evaporating droplet & {\shortstack[l]{Holman Ch2\\Bird Ch10 \& Ch12}}\\ [1ex]
    \hline
    4 (02/13) & a draining cone-shape reservoir, Bernoulli equation, numerical solution, finite difference methods, Euler's method, Newton-Raphson method, projectile with nonlinear drag, a radiating object & {\shortstack[l]{Munson Ch3\\Miller Ch10\\Holman Ch8}}\\ [1ex]
    \hline
    5 (02/20) & 1D flows, fixed points \& stability $+$ project starts & Strogatz Ch2 \\ [1ex]
    \hline
    6 (02/27) & population growth, linear stability analysis, potentials, language death, laser threshold, bifurcations & Strogatz Ch2 \& Ch3\\ [1ex]
    \hline
    7 (03/06) & centrifugal force, overdamped bead on a rotating hoop, insect outbreak & Strogatz Ch3 \\ [1ex]
    \hline
    8 (03/13) & \textbf{No Classes (Spring Break)} & \\ [1ex]
    \hline
    9 (03/20) & 2D flows, simple harmonic oscillator, classification of 2D linear systems, love affairs & Strogatz Ch5 \\ [1ex]
    \hline
    10 (03/27) & regular perturbation theory, projectile motion with nonlinear drag, weakly damped linear oscillator, two-timing  & Strogatz Ch7 \\ [1ex]
    \hline
    11 (04/03) & semi-infinite regions, combination of variables, Stefan problem, film condensation & {\shortstack[l]{Bird Ch12\\Holman Ch9}}\\ [1ex]
    \hline 
    12 (04/10) & angular momentum and fixed axis rotation, moment of inertia, parallel axis theorem &  Kleppner Ch7\\ [1ex]
    \hline
    13 (04/17) & torque and angular momentum, conservation of angular momentum, law of equal areas (Kepler's second law), effective area of a far-off planet & Kleppner Ch7\\ [1ex]
    \hline
    14 (04/24) & central force motion, universal features of central force motion, energy equation and diagrams, energy diagram of planetary motion, perturbed circular orbit & Kleppner Ch10 \\ [1ex]
    \hline
    15 (05/01) & planetary motion, elliptic orbits, Kepler's first and third laws, geostationary orbit, satellite orbit transfer & Kleppner Ch10 \\ [1ex]
    
    \bottomrule
  \end{tabular}
\end{center}
The material discussed during the class sessions are, to some extend, based on the {\large{\textbf{Reading}}} column.

\end{document}
